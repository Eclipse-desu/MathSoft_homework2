\documentclass{ctexart}

\usepackage{graphicx}
\usepackage{amsmath}
\usepackage{amssymb}
\usepackage{amsthm}

\newtheorem{lemma}{引理}[section]
\renewcommand{\proofname}{\textbf{证明}}
\newcommand*{\tran}{^{\mkern-1.5mu\mathsf{T}}}

\title{尝试编写 Shell 脚本}

\author{洪艺中 \\ 数学与应用数学 3190105490}

\begin{document}

\maketitle
在阅读了30--60页之后, 我发现关于 \verb|$| 符号相关的引用格式是最为复杂的, 特别是其后面加括号的时候, 功能各不相同. 但是似乎这三十页中没有相关的 Try It Out, 有点可惜. 其他的 Try It Out 的内容都比较好理解, 于是我就选了一个看起来比较好玩的48页的``Returning a Value''
\section{代码和测试结果} % (fold)
\label{sec:代码和测试结果}
文件 \verb|original.sh| 即为我对书上例子的复现, 具体内容如下:
\begin{verbatim}
#!/bin/bash

yes_or_no() {
  echo "Is your name $* ?"
  while :
  do
    echo -n "Enter yes or no: "
    read x
    case "$x" in
      [yY] | [yY][eE][sS] ) return 0;;
      [nN][oO] ) return 1;;
      * ) echo "Answer yes or no";;
    esac
  done
}

echo "Original parameters are $*"

if yes_or_no "$1"
then
  echo "Hi $1, a good name"
else
  echo "Never mind"
fi

exit 0
\end{verbatim}\par
调用时, 最好额外给出至少一个参数. 如可以输入 \verb|bash original.sh a b| 来启动. 此外, 后面会有一步询问用户, 需要用户输入 \verb|yes| 或者 \verb|no|. 在前面的调用和\verb|yes|的回答下, 输出为
\begin{verbatim}
$ bash original.sh a b
Original parameters are a b
Is your name a ?
Enter yes or no: yes
Hi a, a good name
$
\end{verbatim}\par
接下来具体说明脚本的执行过程.
\begin{enumerate}
	\item 首先会输出 \verb|Original parameters are| 以及用户输入的全部参数, 当用户没有输入任何参数时, 后面也是空字符串.
	\item 之后, 系统会输出 \verb|Is your name| 加上输入的第一个参数 \verb|$1|(当用户没有输入任何参数时, 这里也是空字符串), 也就是询问用户其输入的第一个参数「是不是用户的名字」, 并根据用户的回答给出不同的回答. 如果用户回答的不是 \verb|y|, \verb|yes|, \verb|no| 的其中之一(不区分大小写), 就会输出 \verb|Answer yes or no| 并再次等待输入, 直到输入属于上面的三类.
\end{enumerate}\par
% section 代码和测试结果 (end)
\section{学习心得} % (fold)
\label{sec:学习心得}
这一份脚本中包含了多个值得注意的细节, 下面逐一说明. \par
\subsection{真值的写法} % (fold)
\label{sub:真值的写法}
这个脚本中有一个 \verb|while| 死循环, 书上原版用的是 \verb|while true|, 但是后面它也讲到 \verb|:| 是 ``built-in'' 的, 所以更为高效, 不过这种写法要牺牲可读性. 因此如果对速度要求很高, 应该选用 \verb|:|.
% subsection 真值的写法 (end)
\subsection{函数} % (fold)
\label{sub:函数}
\verb|Shell| 脚本中的函数定义格式和 \verb|c++| 类似. 但是使用上有一些区别: \verb|Shell| 中函数类似于定义了一个子可执行文件, 其调用方式和其他程序, 脚本以及命令类似, 直接按照 \verb|函数名  函数参数列表| 的形式调用, 而非将函数的参数写在括号中. 同时, \verb|Shell| 脚本的函数定义页不需要在括号中写明参数(至少到书上60页是这样的). \par
因此, 取得函数参数的方法和取得程序调用输入的参数也是一样的, \verb|$*| 代表了全部参数列表, \verb|$1|, \verb|$2| 等则对应第 1, 2 个参数.
% subsection 函数 (end)
\subsection{case 结构} % (fold)
\label{sub:verb_case_结构}
\verb|Shell| 脚本中的 \verb|case| 格式和 \verb|c++| 也类似, 只是具体语法不同. 在使用 \verb|case| 的时候, 有一些注意事项和技巧:
\begin{enumerate}
	\item \verb|case| 的同一分支的条件可以写的较为复杂, 可以利用 \verb!|! 分割不同的判定条件. (似乎实际上是用一个正则表达式来判断, 所以功能还可以更丰富?)
	\item \verb|case| 一旦遇到满足的条件, 就会进入执行, 不会判断之后的其他可能更精确的条件. 因此必须把 \verb|*| 这样含义广泛的条件放在后面.
\end{enumerate}
% subsection verb_case_结构 (end)
\subsection{参数调用} % (fold)
\label{sub:参数调用}
一般来说, 如果 \verb|var| 是一个变量, \verb|$var| 和 \verb|"$var"| 的效果是类似的, 都是具体调用其值而非直接获得 \verb|$var| 这个字符串. 但是在上面的脚本中\\
\begin{verbatim}
    case "$x" in
\end{verbatim}
和
\begin{verbatim}
if yes_or_no "$1"
\end{verbatim}
这样的调用全都加了引号. 这也是书上介绍的一个避免\textbf{空输入引发程序错误}的技巧. 如果不加引号, 而且用户输入了空字符串, 那么在具体把 \verb|$var| 转化为其内容是也会得到空的结果, 在有些时候(比如使用 \verb|[]| 这个判断命令时)就会出错.\par
我认为 \verb|$| 引用的机制和 \verb|c++| 中宏定义类似, 是字符串的直接替换. 因此在调用参数值时, 需要考虑字符替换之后是否会引发其他错误. 加上引号之后, 无论结果是否为空, 得到的都是一个带引号括住的字符串, 就不会出现类似问题. 所以加上引号也是一个比较安全的好习惯.
% subsection 参数调用 (end)
% section 学习心得 (end)

\end{document}
